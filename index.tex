\section{Diritto di accesso agli atti}
La proposta di regolamento sulla trasparenza intende:

\begin{itemize}
    \item
        dare pieno compimento
        allo~\href{http://www.alumniscuolagalileiana.it/wp-content/uploads/2017/03/Statuto-Alumni-SGSS.pdf}{statuto}~in
        tema di partecipazione dei soci alle attività sociali e di controllo
        effettivo sulle medesime;
    \item
        formalizzare le procedure di trattamento e conservazione dei dati
        personali, nel rispetto
        del~\href{http://www.garanteprivacy.it/web/guest/home/docweb/-/docweb-display/docweb/1311248}{codice
        in materia di protezione dei dati personali}~(di
        seguito:~\emph{codice}) e
        dell'\href{http://garanteprivacy.it/web/guest/home/docweb/-/docweb-display/docweb/5803310}{autorizzazione
        del garante per la privacy}~(di seguito: \emph{autorizzazione}).
\end{itemize}

In particolare, si ritiene che la~\emph{comunicazione}~\emph{ai
soci}~(\href{http://www.garanteprivacy.it/web/guest/home/docweb/-/docweb-display/docweb/1311248}{codice,
art. 4, comma l}) di documenti associativi contenenti dati personali di
altri soci e delle persone fisiche o giuridiche beneficiarie di
trattamenti economici o dei servizi dell'associazione sia, in seguito
alla sottoscrizione di un'apposita informativa da predisporre, conforme
all'\href{http://garanteprivacy.it/web/guest/home/docweb/-/docweb-display/docweb/5803310}{autorizzazione
(punto 2)}.

Ai fini della trasparenza è indispensabile fornire immediato accesso ai soci a tutti gli
atti degli organi sociali e ai movimenti di cassa, limitando l'accesso solo
in caso di presenza di dati sensibili.
Cio è particolarmente rilevante in quanto i bilanci consolidati approvati in
sede di assemblea rappresentano una visione troppo aggregata delle entrate
e delle uscite prodotte dall'associazione, e non consentono un'adeguata attività
di controllo indipendente in capo ai soci. 

Si rileva che la~\emph{diffusione} all'esterno dell'associazione
(\href{http://www.garanteprivacy.it/web/guest/home/docweb/-/docweb-display/docweb/1311248}{codice,
art. 4, comma m}), nonché il trattamento di dati riguardanti lo stato di
salute e la vita sessuale dei soci (autorizzazione, punto 4) sono
esplicitamente esclusi e illeciti e, per quanto riguarda
l'associazione, valutati dal collegio dei garanti per sanzioni
fino all'espulsione.

Si individua, rispetto all'attuale situazione circa i documenti in
possesso dell'associazione, la seguente organizzazione:

\begin{enumerate}
    \item
        materiale di dominio pubblico, accessibile a chiunque dal sito
        dell'associazione;
    \item
        materiale comunicato immediatamente ai soci, ovvero ad essi
        accessibile attraverso un sistema informatico nel rispetto dei
        requisiti minimi di sicurezza del codice
        (\href{http://www.garanteprivacy.it/web/guest/home/docweb/-/docweb-display/docweb/1311248}{art.
        33 e 34});
    \item
        materiale riservato e comunicabile su richiesta di un socio al
        segretario, su cui delibera la successiva seduta del consiglio di
        amministrazione. Una decisione contraria alla richiesta del socio è
        impugnabile presso il collegio dei garanti;
    \item
        materiale non comunicabile.
\end{enumerate}

\subsection{Materiale di dominio pubblico}\label{materiale-di-dominio-pubblico}

Sono di dominio pubblico lo statuto, i regolamenti di tutti gli organi
associativi, il bilancio consuntivo, e i documenti di proprietà
intellettuale dell'associazione rilasciati nel dominio pubblico.

\subsection{Materiale comunicato ai soci}\label{materiale-comunicato-ai-soci}

Sono automaticamente comunicati ai soci, che si identifichino su un apposito portale ad
accesso controllato con chiavi d'accesso personali:

\begin{enumerate}
    \item
        verbali di tutti gli organi sociali (assemblea, consigli di
        amministrazione, collegio dei garanti, collegio dei revisori dei
        conti), entro quindici giorni dalla loro approvazione. Lo stesso si
        applica non solo al verbale propriamente detto, ma a tutti i suoi
        allegati, delibere e mozioni. Nel caso in cui, in fase di stesura
        o anche di successiva segnalazione, si rilevi la presenza di dati
        riguardanti
        \begin{enumerate}
            \item lo stato di salute o la vita sessuale;
            \item l'origine razziale ed etnica, le
                convinzioni religiose, filosofiche o di altro genere, le opinioni
                politiche, l'adesione a partiti, sindacati, associazioni od
                organizzazioni a carattere religioso, filosofico, politico o
                sindacale
        \end{enumerate}
        i rispettivi passaggi sono sostituiti da~\emph{omissis}. Il verbale in
        forma integrale è comunque conservato (autorizzazione, punto 4, ultimo
        capoverso) e passa in categoria 4 (caso 2.1.a) o categoria 3 (caso
        2.1.b). Tuttavia, non è possibile impedire ai soci l'accesso a
        informazioni contenute nei verbali per ragioni diverse da quelle
        della protezione dei dati personali e sensibili, e in particolare non è
        possibile omettere passaggi per ragioni di opportunità politica o
        economica. Per esempio è vietato conferire incarichi o competenze
        economiche con la condizione che l'identità del beneficiario non sia
        conoscibile ai soci.
    \item
        Dati sintetici ma completi dei flussi di cassa, ovvero una tabella
        annuale ad opera del tesoriere, che contenga per ogni entrata o uscita:
        \begin{itemize}
            \item data del trasferimento;
            \item un dato identificativo del soggetto in rapporto economico
                con l'associazione (nome e cognome per le persone fisiche,
                denominazione dell'ente per le persone giuridiche);
            \item una sintetica descrizione;
            \item per le uscite, un riferimento alla delibera dell'organo
                associativo che le ha autorizzate (per esempio: ``cda
                14/11/2017'').
        \end{itemize}
        Tale tabella deve essere disponibile almeno trenta giorni prima
        dell'assemblea annuale in cui i soci approvano il relativo bilancio.
    \item
        Adesione a registri comunali, provinciali, regionali, agenzia delle
        entrate;
    \item
        relazioni annuali;
    \item
        progetti presentati dall'associazione;
    \item
        convenzioni, contratti, accordi di cooperazione o partenariato
        dell'associazione con enti pubblici o privati;
    \item
        elenco dei soci.
\end{enumerate}

\subsection{Materiale riservato e comunicabile su
richiesta}\label{materiale-riservato-e-comunicabile-su-richiesta}

In osservanza al principio di non eccedenza e indispensabilità della
comunicazione ai soci rispetto al loro diritto di controllo delle
attività associative, si ritiene che non sia necessario rendere
immediatamente consultabili:

\begin{enumerate}
    \item
        contratti stipulati tra l'associazione e singole persone fisiche per
        prestazioni d'opera, collaborazioni o altra natura;
    \item
        tagliandi, fatture, giustificativi di spesa;
    \item
        verbali senza \emph{omissis} posti nel caso descritto al 2.1.b.
    \item
        domande e relativi allegati raccolti in risposta a bandi
        dell'associazione, rivolti a soci o pubblici.
\end{enumerate}

La richiesta di accesso è presentata dal socio al segretario che la
riferisce al primo consiglio di amministrazione utile. Il consiglio di
amministrazione delibera a maggioranza semplice sull'invio al socio del
documento richiesto, cui provvede il segretario.
Una decisione di non comunicabilità dell'atto è appellabile presso il
collegio dei garanti.

\subsection{Materiale non comunicabile}\label{materiale-non-comunicabile}

Questo materiale è disponibile solo ai membri degli organi associativi
esplicitamente incaricati del loro trattamento,
con speciale accordo di riservatezza:

\begin{enumerate}
    \item
        credenziali dei servizi informatici dell'associazione;
    \item
        credenziali per la gestione del conto bancario;
    \item
        dati non anonimizzati del censimento;
    \item
        verbali senza \emph{omissis} posti nel caso descritto al 2.1.a.
\end{enumerate}

\section{Gestione, sicurezza e ciclo di vita dei dati personali}

Il presidente può nominare un responsabile del trattamento dei dati in possesso
dell'associazione.

I documenti dell'associazione sono conservati su un supporto informatico in
cloud (per esempio \emph{Google Drive}) che consenta l'accesso a soci e
collaboratori identificati e autorizzati personalmente, per esempio
attraverso il loro indirizzo email.
Questa misura è necessaria 
\begin{itemize}
    \item per ridurre il rischio di accessi non
autorizzati ai dati dell'associazione, evitando la comunicazione di
credenziali su canali non sicuri;
\item per un'agevole gestione, concessione e revoca delle autorizzazioni;
\item per conservare un registro degli accessi che consenta procedure di
    auditing in caso di violazione dei sistemi informatici, di uso scorretto
    o illegale dei dati personali affidati all'associazione, a tutela anche
    legale dell'associazione.
\end{itemize}

Il responsabile del trattamento dei dati gestisce le credenziali di accesso
all'account dell'associazione in osservanza alle disposizioni
contenute
nell'\href{http://www.garanteprivacy.it/web/guest/home/docweb/-/docweb-display/docweb/1557184}{allegato
B del codice}:
\begin{itemize}
    \item attraverso l'account dell'associazione, autorizza i soci a
        visionare i documenti di cui al~\ref{materiale-comunicato-ai-soci},
        o i membri degli organi, delle commissioni e altri collaboratori
        dell'associazione per i documenti necessari allo svolgimento dei
        compiti loro assegnati;
    \item verifica con cadenza almeno annuale la lista degli 
        autorizzati alla visualizzazione dei documenti di cui al punto
        precedente (codice B, allegato B, punto 15);
    \item segnala al presidente, che riferisce in consiglio di
        amministrazione, eventuali reclami di soci circa il
        trattamento dei propri dati personali, e qualunque fatto o
        comportamento che a suo avviso possa compromettere la sicurezza dei
        dati;
    \item gestisce le richieste dei titolari di dati personali raccolti
        dall'associazione ai sensi dell'articolo 7 del codice;
    \item la parola chiave dell'account dell'associazione, quando è prevista,
        è composta da almeno otto caratteri oppure, nel caso in cui lo
        strumento elettronico non lo permetta, da un numero di caratteri
        pari al massimo consentito; essa non contiene riferimenti
        agevolmente riconducibili al responsabile ed è modificata da
        quest'ultimo al primo utilizzo e, successivamente, almeno ogni sei
        mesi (codice, allegato B, punto 5);
    \item è fatto divieto assoluto al presidente e al responsabile di
        comunicare ad altri le credenziali di autenticazione;
    \item predispone il salvataggio di tutti i documenti con cadenza almeno
        settimanale (codice, allegato B, punto 18);
    \item aggiorna con cadenza almeno annuale il documento programmatico
        sulla sicurezza, che contenga l'identificazione del titolare del
        trattamento (il presidente) e degli eventuali responsabili
        incaricati, e la descrizione degli accorgimenti e delle misure di
        sicurezza adottate per evitare i rischi derivanti dal trattamento di
        dati personali.
\end{itemize}
