\section{Diritto di accesso agli atti}
La proposta di regolamento sulla trasparenza intende:

\begin{itemize}
    \item
        dare pieno compimento
        allo~\href{http://www.alumniscuolagalileiana.it/wp-content/uploads/2017/03/Statuto-Alumni-SGSS.pdf}{statuto}~in
        tema di partecipazione dei soci alle attività sociali e di controllo
        effettivo sulle medesime;
    \item
        formalizzare le procedure di trattamento e conservazione dei dati
        personali, nel rispetto
        del~\href{http://www.garanteprivacy.it/web/guest/home/docweb/-/docweb-display/docweb/1311248}{codice
        in materia di protezione dei dati personali}~(di
        seguito:~\emph{codice}) e
        dell'\href{http://garanteprivacy.it/web/guest/home/docweb/-/docweb-display/docweb/5803310}{autorizzazione
        del garante per la privacy}~(di seguito: \emph{autorizzazione}).
\end{itemize}

In particolare, si ritiene che la~\emph{comunicazione}~\emph{ai
soci}~(\href{http://www.garanteprivacy.it/web/guest/home/docweb/-/docweb-display/docweb/1311248}{codice,
art. 4, comma l}) di documenti associativi contenenti dati personali di
altri soci e delle persone fisiche o giuridiche beneficiarie di
trattamenti economici o dei servizi dell'associazione sia, in seguito
alla sottoscrizione di un'apposita informativa da predisporre, conforme
all'\href{http://garanteprivacy.it/web/guest/home/docweb/-/docweb-display/docweb/5803310}{autorizzazione
(punto 2)}.

Ai fini della trasparenza è indispensabile fornire immediato accesso ai soci a tutti gli
atti degli organi sociali e ai movimenti di cassa, limitando l'accesso solo
in caso di presenza di dati sensibili.
Cio è particolarmente rilevante in quanto i bilanci consolidati approvati in
sede di assemblea rappresentano una visione troppo aggregata delle entrate
e delle uscite prodotte dall'associazione, e non consentono un'adeguata attività
di controllo indipendente in capo ai soci. 

Si rileva che la~\emph{diffusione} all'esterno dell'associazione
(\href{http://www.garanteprivacy.it/web/guest/home/docweb/-/docweb-display/docweb/1311248}{codice,
art. 4, comma m}), nonché il trattamento di dati riguardanti lo stato di
salute e la vita sessuale dei soci (autorizzazione, punto 4) sono
esplicitamente esclusi e illeciti e, per quanto riguarda
l'associazione, valutati dal collegio dei garanti per sanzioni
fino all'espulsione.
