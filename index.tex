La proposta di regolamento sulla trasparenza intende:

\begin{itemize}
    \item
        dare pieno compimento
        allo~\href{http://www.alumniscuolagalileiana.it/wp-content/uploads/2017/03/Statuto-Alumni-SGSS.pdf}{statuto}~in
        tema di partecipazione dei soci alle attività sociali e di controllo
        effettivo sulle medesime;
    \item
        formalizzare le procedure di trattamento e conservazione dei dati
        personali, nel rispetto
        del~\href{http://www.garanteprivacy.it/web/guest/home/docweb/-/docweb-display/docweb/1311248}{codice
        in materia di protezione dei dati personali}~(di
        seguito:~\emph{codice}) e
        dell'\href{http://garanteprivacy.it/web/guest/home/docweb/-/docweb-display/docweb/5803310}{autorizzazione
        del garante per la privacy}~(di seguito: \emph{autorizzazione}).
\end{itemize}

In particolare, si ritiene che la~\emph{comunicazione}~\emph{ai
soci}~(\href{http://www.garanteprivacy.it/web/guest/home/docweb/-/docweb-display/docweb/1311248}{codice,
art. 4, comma l}) di documenti associativi contenenti dati personali di
altri soci e delle persone fisiche o giuridiche beneficiarie di
trattamenti economici o dei servizi dell'associazione sia, in seguito
alla sottoscrizione di un'apposita informativa da predisporre, conforme
all'\href{http://garanteprivacy.it/web/guest/home/docweb/-/docweb-display/docweb/5803310}{autorizzazione
(punto 2)}.

Si rileva che la~\emph{diffusione} all'esterno dell'associazione
(\href{http://www.garanteprivacy.it/web/guest/home/docweb/-/docweb-display/docweb/1311248}{codice,
art. 4, comma m}), nonché il trattamento di dati riguardanti lo stato di
salute e la vita sessuale dei soci (autorizzazione, punto 4) sono
esplicitamente esclusi e sono comportamenti illeciti e, per quanto
riguarda l'associazione, valutati dal collegio dei garanti per sanzioni
fino all'espulsione.

Si individua, rispetto all'attuale situazione circa i documenti in
possesso dell'associazione, la seguente organizzazione:

\begin{enumerate}
    \item
        materiale di dominio pubblico, accessibile a chiunque dal sito
        dell'associazione;
    \item
        materiale comunicato immediatamente ai soci, ovvero ad essi
        accessibile attraverso un sistema informatico nel rispetto dei
        requisiti minimi di sicurezza del codice
        (\href{http://www.garanteprivacy.it/web/guest/home/docweb/-/docweb-display/docweb/1311248}{art.
        33 e 34});
    \item
        materiale riservato e comunicabile su richiesta di un socio al
        segretario, su cui delibera la successiva seduta del consiglio di
        amministrazione. La delibera è impugnabile presso il collegio dei
        garanti;
    \item
        materiale non comunicabile.
\end{enumerate}

\section{Materiale di dominio
pubblico}\label{materiale-di-dominio-pubblico}

Sono di dominio pubblico lo statuto, i regolamenti di tutti gli organi
associativi, e i documenti di proprietà intellettuale dell'associazione
rilasciati nel dominio pubblico.

\section{Materiale comunicato ai
soci}\label{materiale-comunicato-ai-soci}

Sono comunicati ai soci, che si identifichino su un apposito portale ad
accesso controllato secondo le misure di sicurezza citate:

\begin{enumerate}
    \item
        verbali di tutti gli organi sociali (assemblea, consigli di
        amministrazione, collegio dei garanti, collegio dei revisori dei
        conti). Nel caso in cui, in fase di stesura o anche di successiva
        segnalazione, si rilevi la presenza di dati riguardanti
        \begin{enumerate}
            \item lo stato di salute o la vita sessuale;
            \item l'origine razziale ed etnica, le
                convinzioni religiose, filosofiche o di altro genere, le opinioni
                politiche, l'adesione a partiti, sindacati, associazioni od
                organizzazioni a carattere religioso, filosofico, politico o
                sindacale.
        \end{enumerate}
        i rispettivi passaggi sono sostituiti da~\emph{omissis}. Il verbale in
        forma integrale è comunque conservato (autorizzazione, punto 4, ultimo
        capoverso) e passa in categoria 4 (caso 2.1.a) o categoria 3 (caso
        2.1.b). Tuttavia, non è possibile impedire ai soci l'accesso a
        informazioni contenute nei verbali da parte dell'organo sociale
        coinvolto per ragioni diverse da quelle della protezione dei dati
        personali, e in particolare non è possibile omettere passaggi per
        ragioni di opportunità politica o economica. Per esempio è vietato
        conferire incarichi o competenze economiche con la condizione che
        l'identità del beneficiario non sia conoscibile ai soci.
    \item
        Dati sintetici ma completi dei flussi di cassa, ovvero una tabella
        annuale ad opera del tesoriere e approvata dal collegio dei revisori
        dei conti, che contenga per ogni entrata o uscita:
        \begin{itemize}
            \item data del trasferimento;
            \item un dato identificativo del soggetto in rapporto economico con l'associazione (nome e cognome per le persone fisiche, o denominazione dell'ente per le persone giuridiche);
            \item una sintetica descrizione (per esempio: ``rimborso attività fare ricerca'', ``quota associativa'', ``donazione'', ``fondi fare ricerca'');
            \item per le uscite, un riferimento alla delibera dell'organo associativo che le ha autorizzate (per esempio: ``cda 14/11/2017'').
        \end{itemize}
        cio è particolarmente rilevante in quanto i bilanci aggregati
        approvati in sede di assemblea rappresentano una visione troppo
        grossolana delle entrate e delle uscite prodotte dall'associazione, e
        non consentono alcuna attività di controllo indipendente in capo ai
        soci.
    \item
        Adesione a registri comunali, provinciali, regionali, agenzia delle
        entrate;
    \item
        relazioni annuali;
    \item
        progetti presentati dall'associazione;
    \item
        convenzioni, contratti, accordi di cooperazione o partenariato
        dell'associazione con enti pubblici o privati;
    \item
        elenco dei soci.
\end{enumerate}

\section{Materiale riservato e comunicabile su
richiesta}\label{materiale-riservato-e-comunicabile-su-richiesta}

In osservanza al principio di non eccedenza e indispensabilità della
comunicazione ai soci rispetto al loro diritto di controllo delle
attività associative, si ritiene che non sia necessario rendere
immediatamente consultabili:

\begin{enumerate}
    \item
        contratti stipulati tra l'Associazione e singole persone fisiche per
        prestazioni d'opera, collaborazioni o altra natura;
    \item
        tagliandi, fatture, giustificativi di spesa;
    \item
        verbali senza \emph{omissis} posti nel caso descritto al 2.1.b.
\end{enumerate}

La richiesta presentata dal segretario per conto del socio in occasione
del primo consiglio di amministrazione utile è deliberata dal consiglio
di amministrazione. Una decisione di non comunicabilità dell'atto è
appellabile presso il collegio dei garanti.

\section{Materiale non comunicabile}\label{materiale-non-comunicabile}

Questo materiale è disponibile solo ai membri degli organi associativi,
con speciale accordo di riservatezza e misure supplementari
crittografiche di sicurezza:

\begin{enumerate}
    \item
        credenziali dei servizi informatici dell'associazione;
    \item
        credenziali per la gestione del conto economico;
    \item
        dati non anonimizzati del censimento;
    \item
        verbali senza \emph{omissis} posti nel caso descritto al 2.1.a.
\end{enumerate}

