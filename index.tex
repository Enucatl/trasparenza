Il presente regolamento descrive la procedura di accesso alla
documentazione delle attività dell'Associazione Alumni prodotta dai suoi
organi ed associati nello svolgimento delle proprie attività (di seguito:
regolamento). In tal senso, esso intende dare compimento
allo \href{http://www.alumniscuolagalileiana.it/wp-content/uploads/2017/03/Statuto-Alumni-SGSS.pdf}{Statuto
dell'Associazione Alumni} (di seguito: statuto) in tema di partecipazione
dei soci ordinari regolarmente iscritti all'Associazione (di seguito: soci)
alle attività sociali e di controllo effettivo sulle medesime, nonché
formalizzare le procedure di trattamento e conservazione dei dati
personali, nel rispetto del \href{http://www.garanteprivacy.it/web/guest/home/docweb/-/docweb-display/docweb/1311248}{codice in materia di protezione dei dati
personali}~(di seguito: codice) e dell'\href{http://garanteprivacy.it/web/guest/home/docweb/-/docweb-display/docweb/5803310}{autorizzazione del garante per la
privacy}~(di seguito: autorizzazione). 

\subsection{Livelli e procedure di accesso agli atti}

Si individuano quattro livelli di organizzazione e di accesso relativi alle
diverse forme di documentazione dell'attività dell’Associazione Alumni:

\subsubsection{Materiale di dominio pubblico}\label{materiale-di-dominio-pubblico}

Si tratta di documenti accessibili a chiunque dal sito internet ufficiale
dell'Associazione Alumni. A questa categoria appartengono in particolare: 
\begin{enumerate}
\item statuto;
\item composizione degli organi sociali;
\item bilanci consuntivi;
\item relazioni annuali;
\item elenco dei nominativi dei soci.
\end{enumerate}

\subsubsection{Materiale immediatamente accessibile ai soci}\label{materiale-comunicato-ai-soci}

Si tratta di documenti comunicati immediatamente ai soci ed accessibili
esclusivamente ad essi, dopo essersi identificati con chiavi d'accesso
personali, attraverso un sistema informatico ad accesso controllato, nel
rispetto dei requisiti minimi di sicurezza del codice (art. 33 e 34).
Presupposto per l'accesso agli atti è l'impegno da parte dei soci a non
diffondere in nessun modo la documentazione.  

A questa categoria appartengono in particolare:

\begin{enumerate}
    \item verbali di tutti gli organi sociali (Assemblea dei soci, Consiglio
        di Amministrazione, Collegio dei garanti, Collegio dei revisori dei
        conti), da condividere entro quindici giorni dalla loro
        approvazione. Questo si applica non solo al verbale propriamente
        detto, ma anche a tutti i suoi allegati, delibere, mozioni.
        Nel caso in cui, in fase di stesura o anche di successiva segnalazione, si
        rilevi la presenza di dati riguardanti lo stato di salute o la vita
        sessuale, l'origine razziale ed etnica, le convinzioni religiose,
        filosofiche o di altro genere, le opinioni politiche, l'adesione a partiti,
        sindacati, associazioni od organizzazioni a carattere religioso, filosofico,
        politico o sindacale dei soci, i rispettivi passaggi sono sostituiti
        da \emph{omissis}. Il verbale in forma integrale è tuttavia conservato
        (autorizzazione, punto 4, ultimo capoverso) e passa in modalità di accesso
        4;
    \item il bilancio consuntivo posto in approvazione in Assemblea a seguito
        del parere positivo dei Revisori dei Conti e la relativa
        documentazione comprovante la gestione della cassa sociale
        nell'esercizio di riferimento. Tale documentazione deve riportare
        una descrizione sintetica ma comprensibile di ciascuna entrata e uscita
        dal conto corrente dell'associazione;
    \item regolamenti prodotti dagli organi sociali dell'Associazione
        nell'esercizio delle proprie funzioni, per esempio Regolamento Quote
        Associative, Regolamento Collana, Regolamento Rimborsi Spese,
        Regolamento per la trasparenza e l'accesso agli atti associativi;
    \item certificati di adesione a registri comunali, provinciali, regionali;
    \item certificati di registro presso l'Agenzia delle entrate e del Territorio;
    \item convenzioni, contratti, accordi di cooperazione o parternariato
        dell'Associazione con enti pubblici e privati. Dati personali, per
        esempio firme in originale e indirizzi, non rilevanti ai fini
        del controllo dei soci sull'attività svolta sono oscurati. Il
        documento in versione integrale è conservato in modalità di
        accesso 4; 
    \item elenco completo dei soci; 
    \item pubblicazioni o altra documentazione per cui la proprietà
        intellettuale è stata ceduta interamente all'Associazione.
\end{enumerate}

\subsubsection{Materiale riservato e comunicabile su
richiesta}\label{materiale-riservato-e-comunicabile-su-richiesta}

Si tratta di documenti comunicati previa richiesta presentata da un socio in
forma semplice al Segretario dell'Associazione (o suo delegato), il quale
può o condividere immediatamente il documento richiesto o rimandare la
decisione circa la condivisione in forma motivata al Consiglio
d'Amministrazione. Una decisione del Consiglio d'Amministrazione contraria
all'accesso al documento richiesto può essere impugnata dal socio
richiedente in forma motivata presso il Collegio dei Garanti. Presupposto
per l'accesso agli atti è l'impegno da parte dei soci a non diffondere in
nessun modo la documentazione.  

In osservanza al principio di non eccedenza e indispensabilità della
comunicazione ai soci rispetto al loro diritto di controllo delle attività
associative, nel caso il documento in questione contenga dati personali
o sensibili, per esempio firme in originale e indirizzi personali, non
rilevanti ai fini del controllo dei soci sull'attività svolta, essi sono
oscurati. Il documento integrale è conservato in modalità di accesso 4.

\begin{enumerate}
    \item tagliandi, fatture, giustificativi di spesa;
    \item progetti o altra documentazione relativa ad attività ritenute di
        interesse generale o strategico dal Consiglio di Amministrazione; 
    \item logo e materiali di identità visiva dell'Associazione; 
    \item\label{bandi} domande e relativi allegati raccolti in risposta a
        bandi dell'Associazione Alumni rivolti a soci o pubblici;
    \item\label{contratti} contratti stipulati tra l'Associazione Alumni e
        singole persone fisiche per prestazioni d'opera, collaborazioni o
        altra natura; 
    \item\label{progetti} documenti frutto di una attività di ideazione,
        progettazione, sviluppo o elaborazione di contenuti originali da
        parte dei soci, come ad esempio progetti, studi, ricerche,
        rendiconti, relazioni, consulenze relative ad attività scientifica,
        culturale, professionale, divulgativa o in ogni caso conforme alle
        finalità associative definite dello Statuto, indifferentemente dal
        fatto che questa attività sia stata effettivamente realizzata oppure
        soltanto progettata; 
    \item\label{pubblicazioni} pubblicazioni della Collana Associativa le
        cui condizioni editoriali non implichino una proprietà intellettuale
        esclusiva dei soci coinvolti nell'attività.
\end{enumerate}
Nei casi~\ref{bandi},~\ref{contratti},~\ref{progetti} e~\ref{pubblicazioni} il
Segretario (o suo delegato) condivide la documentazione informando
contestualmente le persone coinvolte. 

\subsubsection{Materiale non comunicabile}\label{materiale-non-comunicabile}

Questo materiale è disponibile solo ai membri degli organi associativi
esplicitamente incaricati del loro trattamento, i quali stipulano speciale
accordo di riservatezza al momento del loro insediamento:

\begin{enumerate}
    \item credenziali per la gestione del conto bancario dell'Associazione;
    \item dati disaggregati del censimento e ulteriori questionari interni
        all'Associazione;
    \item verbali degli organi associativi in versione integrale;
    \item pubblicazioni di proprietà intellettuale esclusiva di soci e
        frutto indiretto dell'attività associativa (qualora condivisi dai
        diretti interessati).
\end{enumerate}

\section*{Gestione, sicurezza e ciclo di vita dei dati personali}
Il Presidente dell'Associazione Alumni può nominare un responsabile del
trattamento dei dati in possesso dell'Associazione Alumni. 

Onde favorire una efficiente gestione degli accessi nonché facilitare la
tutela legale dell'Associazione in caso di abusi, i documenti
dell'Associazione sono conservati su un supporto informatico in cloud
attivato ad hoc (di seguito: account), il quale consenta l'accesso a soci e
collaboratori identificati ed autorizzati personalmente attraverso
l'indirizzo e-mail con cui sono accreditati presso l'Associazione.

Il responsabile del trattamento dei dati gestisce le credenziali di accesso
all'account dell'Associazione Alumni in osservanza alle disposizioni
contenute nell'\href{http://www.garanteprivacy.it/web/guest/home/docweb/-/docweb-display/docweb/1557184}{allegato B del codice}, ed in particolare: 
\begin{enumerate}
    \item garantisce l'accesso ai soci, ai membri degli organi sociali o
        altri collaboratori dell'Associazione relativamente ai documenti
        necessari allo svolgimento dei loro compiti nel quadro delineato dal
        presente Regolamento;
    \item verifica con cadenza almeno annuale la lista degli autorizzati
        alla visualizzazione dei documenti di cui al punto precedente;
    \item riferisce tempestivamente in Consiglio di Amministrazione a
        proposito di eventuali reclami di soci circa il trattamento dei
        propri dati personali e di qualunque fatto o comportamento che a suo
        avviso possa compromettere la sicurezza dei dati dell'Associazione;
    \item gestisce le richieste dei titolari di dati personali raccolti
        dall'associazione ai sensi dell'articolo 7 del codice.
\end{enumerate}
 
La parola chiave dell'account dell'associazione è composta da
almeno otto caratteri oppure, nel caso in cui lo strumento elettronico non
lo permetta, da un numero di caratteri pari al massimo consentito; essa non
contiene riferimenti agevolmente riconducibili al responsabile ed è
modificata da quest'ultimo al primo utilizzo e, successivamente, almeno ogni
sei mesi. È fatto divieto al Presidente ed al responsabile del trattamento
dei dati di comunicare ad altri le credenziali di autenticazione.
